\documentclass[14pt]{beamer}

\mode<presentation>
\usetheme[progressbar=frametitle]{metropolis}

\usepackage{booktabs}
\usepackage[scale=2]{ccicons}
\usepackage{xspace}
\usepackage{xepersian}

\defbeamertemplate*{title page}{customized}[1][]
{	
	\usebeamercolor[fg]{titlegraphic}\inserttitlegraphic\par
	\raggedleft\usebeamerfont{title}\inserttitle\par
	\usebeamerfont{subtitle}\usebeamercolor[fg]{subtitle}\insertsubtitle\par
	\bigskip
	\usebeamerfont{author}\insertauthor\par
	\usebeamerfont{institute}\insertinstitute\par
	\usebeamerfont{date}\insertdate\par
}	

\title{تعیین ساختار ارائه و مطالعه منابع، ابزار مطالعه}
\subtitle{\color{black} روش تحقیق و گزارش نویسی}
\date{\today}
\author{سیاوش کاوسی}
\institute{دانشگاه صنعتی امیرکبیر}
\titlegraphic{\hfill\includegraphics[height=1.5cm]{logo}}

\makeatletter
\newcommand{\rtlist}{\raggedleft\rightskip\@totalleftmargin} 
\makeatother
\newcommand{\sectionfontsize}{\fontsize{22pt}{0pt}\selectfont}
\newcommand{\framefontsizelarge}{\fontsize{18pt}{0pt}\selectfont}
\newcommand{\frametitlefontsize}{\fontsize{20pt}{0pt}\selectfont}
\newcommand{\defaultvspace}{\vspace{5mm}}

% xepersian font settings 
\settextfont{Adobe Arabic}\fontsize{12pt}{0pt}


\begin{document}
\begin{persian}
	\maketitle
	\everypar{\rightskip\rightmargin}		
	
	\section{\sectionfontsize تنظیم ساختار اولیه، مطالعه و یادداشت برداری}	
	
	\begin{frame}{\frametitlefontsize تنظیم ساختار اولیه، مطالعه و یادداشت برداری}
		\framefontsizelarge
		مطالعه و یادداشت برداری نکات مفید منابع \defaultvspace\\
		جهت دادن مطالعه به کمک ساختار اولیه
	\end{frame}	
	
	\begin{frame}{\frametitlefontsize ساختار اولیه}
		منظور، ارائه لیستی از عناوین \defaultvspace\\
		بخش بندی پایه ارائه
		\begin{itemize}\rtlist
			\item مقدمه
			\begin{itemize}\rtlist
				\item شرح موضوع ارائه
			\end{itemize}
			
			\item بدنه اصلی
			\begin{itemize}\rtlist
				\item انتخاب نکات اصلی بحث\\
				\item تعیین محل هر بحث با استفاده از نکات مباحث اصلی و فرعی
				\item ساده نبودن تعیین نکات اصلی و فرعی
				\item بزرگنمایی بیش از حد مباحث غیر مهم
			\end{itemize}
			
			\item نتیجه گیری
		\end{itemize}
	\end{frame}
	
	\begin{frame}{\frametitlefontsize ساختار اولیه - ادامه}
		مثال مربوطه - صورت اول
		\begin{enumerate}\rtlist
			\item روش های آستانه یابی تکراری
			\\ ... (شرح مطلب)
			\begin{enumerate}\rtlist
				\item روش آستانه یابی کیتلر و ایلینگورذ
				\\ ... (شرح مطلب)
			\end{enumerate}
			
			\item روش های آستانه یابی غیرتکراری
			\begin{enumerate}\rtlist
				\item روش های آستانه یابی بر پایه آنتروپی
				\\ ... (شرح مطلب)
				\begin{enumerate}\rtlist
					\item روش آنتروپی شانون
					\\ ... (شرح مطلب)
				\end{enumerate}
				
				\item روش آستانه یابی بر پایه هیستوگرام
			\end{enumerate}
		\end{enumerate}
	\end{frame}
	
	\begin{frame}{\frametitlefontsize ساختار اولیه - ادامه}
		مثال مربوطه - صورت دوم
		\begin{enumerate}\rtlist
			\item روش های آستانه یابی تکراری کیتلر و ایلینگورذ
			\\ ... (شرح مطلب)
			
			\item روش های آستانه یابی غیرتکراری
			\begin{enumerate}\rtlist
				\item روش آستانه یابی بر پایه آنتروپی شانون
				\\ ... (شرح مطلب)
				
				\item روش آستانه یابی بر پایه هیستوگرام
				\\ ... (شرح مطلب)				
			\end{enumerate}
		\end{enumerate}
	\end{frame}
	
	\begin{frame}{\frametitlefontsize ساختار اولیه - ادامه}
		مشخص کردن ترتیب قرار گرفتن مباحث پس از تعیین مباحث اصلی و با توجه به نکات زیر
		
		\begin{itemize}\rtlist
			\item جریان ایده ها از مقدمه شروع و باید با سیر منطقی به نتایج برسد
			\item هر بخش باید زمینه ساز و تسهیل کننده درک بخش بعدی باشد 
			\item تقدم و تأخر مطالب باید به گونه ای باشد که مخاطب بتواند مطالب را به راحتی دنبال کند 
			\item ترتیب مباحث ممکن است تحت تاثیر ماهیت موضوع قرارگیرد
		\end{itemize}
	\end{frame}
	
	\begin{frame}{\frametitlefontsize ساختار اولیه - ادامه}
		مثال های مربوط به ترتیب مباحث
		
		ترتیب زمانی: توالی مطالب توالی زمانی است \\
		ترتیب حل مشکل: طرح مشکل و جدیت آن سپس ارائه راه حل \\
		بررسی علت و معلولی: ابتدا علل و سپس معلول ها \\ 
		ترتیب افزایش اهمیت: کل به جز یا جز به کل \\ 
		ترتیب موضوعی: تقسیم به چند زیرموضوع که هر یک بیانگر یک نکته اصلی 
	\end{frame}
	
	\begin{frame}{\frametitlefontsize مطالعه و یادداشت برداری}
		\framefontsizelarge
		جهت دادن به مطالعه با استفاده از یادداشت برداری مطالب مفید هر بخش، در واقع جهت دار بودن مطالعه از اتلاف وقت جلوگیری می کند
	\end{frame}

	\begin{frame}{\frametitlefontsize مطالعه مقالات علمی}
		در اغلب موارد لازم است که برای فهمیدن یک مقاله چند بار آن را خواند
		
		دور اول برای اجتناب از گیجی می توان مطالب مربوط به مرور سوابق موضوع و توسعه روابط ریاضی را نخواند\\
		دور دوم همه مطالب خوانده می شوند\\
		دور سوم هم مطالب درک نشده در دور قبل بهتر فهمیده می شوند\\
		در متون فنی واژه ها و یا ترکیب آن ها مفاهیم خاصی را می رسانند که باید از متون تخصصی پیدا نمود
		
	\end{frame}
	
	\begin{frame}{\frametitlefontsize مطالعه مقالات علمی - ادامه}
		برای فهم یک مقاله باید مطالب زیر روشن شود :\\
		\begin{itemize}\rtlist
			\item مسأله مورد بررسی مقاله
			\item قدم های اصلی راه حل
			\item چگونگی پیاده سازی راه حل
		\end{itemize}
	\end{frame}
	
	\begin{frame}{\frametitlefontsize مطالعه مقالات علمی - ادامه}
		دانستن اینکه هر بخش شامل چه مطالبی می شود باعث افزایش کارایی می شود\\
		
		\begin{itemize}\rtlist
			\item بخش مقدمه انگیزه انجام کار و روش کلی حل مساله
			\item بدنه اصلی مقاله راه حل و ارزیابی آن را با جزئیات زیاد ارائه می کند
			\item بخش سوابق موضوع کارهای مرتبط موجود را مرور می کند
			\item بخش نتیجه گیری خلاصه آورده های مقاله
		\end{itemize}
	\end{frame}

	\begin{frame}{\frametitlefontsize مطالعه مقالات علمی - ادامه}
		\textbf{\large{نکات قابل ذکر در حین مطالعه مقاله}}\newline\newline
		نکات مهم باید با رنگ آمیزی برجسته شوند\\
		در اول هر پاراگراف نوشتن یکی از کلمات \textbf{انگیزه/مساله}، \textbf{ایده/راه جل}، \textbf{ارزیابی و آورده}\\
		اگر پژوهشگر سوالی در مورد هر پاراگراف داشت، باید آن را در حاشیه پاراگراف نوشت\\
		سوالات اساسی را باید در بالای صفجه یا در انتهای مقاله نوشت\\
		سرانجام جمع بندی کلی پژوهشگر از مقاله باید در بالای صفحه اول مقاله نوشته شود\\
	\end{frame}
	
	\begin{frame}{\frametitlefontsize راهکار الکترونیکی بر مطالعه مقالات}
		\framefontsizelarge
		در فاز مطالعه می توان از کامپیوتر و نرم افزار های موجود استفاده کرد \defaultvspace\\
		امکانات خوبی برای یادداشت برداری و برجسته کردن و ... را در اختیار داریم \\
	\end{frame}

	\begin{frame}{\frametitlefontsize ویرایش متون و حاشیه نویسی در نرم افزارهای ورد و آکروبات}
		\framefontsizelarge
		امروزه اکثر مستندات الکترونیکی در قالب فایل های پی دی اف هستند \defaultvspace\\
		پی دی اف در نرم افزار هائی مثل ورد یا اکروبات تولید می شود \defaultvspace\\
		\textbf{امکاناتی که اینگونه نرم افزارها در اختیار ما گذاشته اند}
		\begin{itemize}\rtlist
			\item اعلام نظر
			\item حاشیه نویسی 
			\item اصلاح اشکالات
		\end{itemize}
	\end{frame}
	
	\begin{frame}{\frametitlefontsize امکانات نرم افزارهای ورد - آکروبات}
		\textbf{مایکروسافت ورد}
		\begin{itemize}\rtlist
			\item ویرایش متن
			\item بررسی فایل ویرایش شده
		\end{itemize}
		
		\textbf{آکروبات}
		\begin{itemize}\rtlist
			\item ویرایش متن
			\begin{itemize}\rtlist
				\item یادداشت چسبان
				\item امکانات ویرایشی
				\item رنگ آمیزی زمینه
				\item مهر زدن
			\end{itemize}
			\item بررسی فایل ویرایش شده
		\end{itemize}		
	\end{frame}
	
\end{persian}
\end{document}