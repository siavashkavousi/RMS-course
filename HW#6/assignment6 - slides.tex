\documentclass[14pt]{beamer}

\mode<presentation>
\usetheme[progressbar=frametitle]{metropolis}

\usepackage{booktabs}
\usepackage[scale=2]{ccicons}
\usepackage{xspace}
\usepackage{xepersian}

\defbeamertemplate*{title page}{customized}[1][]
{	
	\usebeamercolor[fg]{titlegraphic}\inserttitlegraphic\par
	\raggedleft\usebeamerfont{title}\inserttitle\par
	\usebeamerfont{subtitle}\usebeamercolor[fg]{subtitle}\insertsubtitle\par
	\bigskip
	\usebeamerfont{author}\insertauthor\par
	\usebeamerfont{institute}\insertinstitute\par
	\usebeamerfont{date}\insertdate\par
}	

\title{یادداشت برداری و ابزار آن، انجام بخش عملی پژوهش و نوشتن نسخه اولیه گزارش}
\subtitle{\color{brown} روش تحقیق و گزارش نویسی}
\date{17 فروردین 1395}
\author{سیاوش کاوسی}
\institute{دانشگاه صنعتی امیرکبیر}
\titlegraphic{\hfill\includegraphics[height=1.5cm]{logo}}

\makeatletter
\newcommand{\rtlist}{\raggedleft\rightskip\@totalleftmargin} 
\makeatother
\newcommand{\sectionfontsize}{\fontsize{22pt}{0pt}\selectfont}
\newcommand{\framefontsizelarge}{\fontsize{18pt}{0pt}\selectfont}
\newcommand{\frametitlefontsize}{\fontsize{20pt}{0pt}\selectfont}
\newcommand{\defaultvspace}{\vspace{5mm}}

% xepersian font settings 
\settextfont{Adobe Arabic}\fontsize{12pt}{0pt}


\begin{document}
\begin{persian}
	\maketitle
	\everypar{\rightskip\rightmargin}		
	
	\section{\sectionfontsize یادداشت برداری و ابزار آن}	
	
	\begin{frame}{\frametitlefontsize یادداشت برداری}
		\framefontsizelarge
		تسهیل یادداشت برداری با کمک ساختار تهیه شده در زمان مطالعه منابع و نوشتن روی آنها \defaultvspace\\
		انجام یادداشت برداری بر روی \textbf{فیش}\defaultvspace\\
		فیش ها می توانند هر اندازه ای باشند  \textbf{اما} بهتر است اندازه  $15*10$ یا  $21*15$ سانتیمتر انتخاب شود که اولی برای یادداشت های کوتاه تر و دومی برای یادداشت های بلند تر
	\end{frame}	
	
	\begin{frame}{\frametitlefontsize یادداشت برداری - مثالی از فیش}
		\framefontsizelarge
		\begin{figure}
			\includegraphics[width=\linewidth]{pic.jpg}
		\end{figure}
		عکس مربوطه!!!
	\end{frame}	
	
	\begin{frame}{\frametitlefontsize یادداشت برداری - ساختار فیش}
		\framefontsizelarge
		
		\begin{itemize}\rtlist
			\item در سمت چپ بالا، ذکر نوع یادداشت  
			\defaultvspace\\
			\item در پایین، ذکر نکاتی که پژوهشگر مایل است در مورد مطلب روی فیش فراموش نشود. توصیه می شود مطالبی که ناگهان به ذهن می رسد و چند لحظه بعد فراموش می شود در این مکان نوشته شود
		\end{itemize}
	\end{frame}	
	
	\begin{frame}{\frametitlefontsize یادداشت برداری - نکات مهم در مورد فیش}
		\framefontsizelarge
		روی هر فیش تنها یک مطلب باید نوشته شود نه بیشتر\\
	   نوشتن بیش از یک مطلب روی یک فیش مرحله بعد، دسته بندی یادداشت ها را با دشواری روبرو خواهد ساخت\defaultvspace\\
	   
	   اگر مطلب تا آخر یک فیش به پایان نرسد، در انتهای این فیش و ابتدای فیش بعدی از یک علامت پیکان افقی 
	   ($\leftarrow$)
	   استفاده می شود تا ادامه دار بودن مطلب مشخص گردد	
	\end{frame}
	
	\begin{frame}{\frametitlefontsize یادداشت برداری - انواع}
		\framefontsizelarge
		\begin{itemize}\rtlist
			\item نقل مستقیم
			\item نقل غیر مستقیم
			\item ترجمه با یا بدون خلاصه کردن
			\item یادداشت برداری
		\end{itemize}
	\end{frame}
	
	\begin{frame}{\frametitlefontsize یادداشت برداری - نقل مستقیم}
		\framefontsizelarge
		یادداشت برداری عین متن با همه تاکیدات آن\defaultvspace\\
		در زمان یادداشت برداری از متون دقیق مانند دستورالعمل ها، قوانین و امثال آن لازم می شود، یا وقتی مطلب بسیار خوب بیان شده و هرگونه تغییر در آن، آنرا خدشه دار می کند\defaultvspace\\
		خلاصه کردن نقل مستقیم با حذف قسمت هایی از آن و گذاشتن سه نقطه به جای آن انجام می شود
	\end{frame}
	
	\begin{frame}{\frametitlefontsize یادداشت برداری - نقل غیر مستقیم}
		\framefontsizelarge
	   عین کلمات و عبارات متن اصلی اهمیتی برای پژوهشگر ندارد لذا برای هماهنگ ساختن نوشته ها پژوهشگر ترجیح می دهد مطلب را با بیان خود بازگو کند\defaultvspace\\
		پژوهشگر باید دقت داشته باشد تا مفاهیم متن در بازگوئی تغییر داده نشود\defaultvspace\\
		نقل غیرمستقیم همراه با تلخیص، پژوهشگر باید از ذکر مثال ها و شرح صرف نظر نمود و ایده های اصلی متن را با زبان خود نوشت
	\end{frame}
	
	\begin{frame}{\frametitlefontsize یادداشت برداری - نکات مهم}
		\framefontsizelarge
		یادداشت برداری باید در حدی کامل باشد که پژوهشگر را از مراجعه مجدد به مرجع بی نیاز کند\\
		این قدم گرچه زمان بر است اما در انجام مطلوب تر کار بسیار موثر است و با تسریعی که در قدم های بعدی ایجاد می کند مجموعا اتلاف وقت ببار نمی آورد\defaultvspace\\
		توصیه می شود که مطالب را با خط کشیدن انتخاب نکنیم اما اگر مطلب طولانی باشد می توان کپی مطلب را تهیه نمود و ضمیمه فیش های مربوطه کرد
	\end{frame}
	
	\begin{frame}{\frametitlefontsize یادداشت برداری - راهکارهای الکترونیکی}
		\framefontsizelarge
		اگر مطالعه با راهکارهای الکترونیکی انجام شده باشد، طبیعی است که یادداشت برداری نیز با همین رویکرد انجام پذیرد. \defaultvspace\\
		می توان از نرم افزارهای واژه پرداز یا نرم افزارهای یادداشت برداری برای تسریع کار استفاده کرد (مثلا نرم افزار وان نوت)
	\end{frame}
	
	\begin{frame}{\frametitlefontsize  یادداشت برداری - دسته بندی و ارزیابی }
		\framefontsizelarge
		دسته بندی فیش ها بر حسب بخش ها پس از مطالعه و یادداشت برداری از همه منابع
		
		بررسی و تصمیم گیری در مورد تناسب مجموعه اطلاعات گردآوری شده برای هر بخش و اهمیت آن بخش

		یادداشت برداری از منابع جدید در جهت کاهش نقصان بخش هایی که مطالب کافی بدست نیامده
		
		اگر منابع مناسب پیدا نشود، تنها راه حل تغییر طرح و ساختار ارائه و حذف مباحث مورد بحث است
		
		ارزیابی ساختار ارائه شده و مطالب تهیه شده برای هر بخش و \textbf{تغییر} در صورت لزوم
	\end{frame}
	
	\section{آشنائی با نرم افزار وان نوت}
	
	\begin{frame}{\frametitlefontsize  آشنائی با نرم افزار وان نوت }
		\framefontsizelarge
		برای ثبت سریع ، جمع آوری ، سازماندهی ، و به اشتراک گذاشتن اطلاعات ایجاد گردیده است 
		
		اطلاعات در پنج سطح دفتر ، گروه ، بخش ، صفحه ، و زیر صفحه قراد می گیرد .
		
		مقایسه وان نوت با کلاسور 
		\begin{itemize}\rtlist
			\item کلاسور متناظر با یک دفتر است 
			\item جدا کننده های کلاسور متناظر با بخش هستند 
			\item بخش های موجود در جدا کننده های متناظر با گروه هستند 
			\item هر بخش به یک درس اختصاص می یابد که متناظر با یک صفحه وان نوت است 
			\item  در صورت تنوع می توان زیر صفحه در آن قرار داد 
		\end{itemize}
	\end{frame}
	
	\begin{frame}{\frametitlefontsize  آشنائی با نرم افزار وان نوت }
		\framefontsizelarge
		وان نوت امکانات جالبی برای ثبت سریع و موقت انواع مطالب ارائه می دهد .
		
		 کاربر در هر محیط یا برنامه ای باشد می تواند با فشرده win + N یک صفحه ی وان نوت باز کرده و مطالب خود را در آن یادداشت کند 
		 
		 ابزار \lr{New Shared Notebook} یک دفتر به اشتراک گذاشته شده را ایجاد می کند.
		 
		 برای سهولت در امر خواندن می توان قسمتی از متن را با عنوان خوانده نشده (Unread) و یا خوانده شده علامت گذاری کرد  
	\end{frame}
	
	\begin{frame}{\frametitlefontsize  آشنائی با نرم افزار وان نوت }
		\framefontsizelarge
		می توان تنظیمات نرم افزار را در گزینه ی Options تغییر داد 
		
		انتخاب Display لیستی از گزینه ها را نمایش می دهد 
		
		انتخاب Proofing مربوط به بخش تشخیص و تصحیح املاء مطالب است 
		
		انتخاب Video and Audio دستگاه های ورودی صوت و تصویر را مشخص می کند 
		
		انتخاب Backup and Save امکاناتی را برای تهیه فایل های پشتیبان در اختیار قرار می دهد  
	\end{frame}
	
	\section{\sectionfontsize انجام بخش عملی پژوهش}	

	\begin{frame}{\frametitlefontsize  انجام بخش عملی پژوهش - مقدمه }
		\framefontsizelarge
		پس از مطالعه پیشینه موضوع و تامل و کاوش در آن، انجام بخش عملی پژوهش در دستور کار قرار می گیرد تا مساله به نحوی قابل قبول و یا بهتر از روش های موجود حل شود
	\end{frame}
	
	\begin{frame}{\frametitlefontsize  انجام بخش عملی پژوهش - تسهیل اجرای پژوهش }
		\framefontsizelarge
		پس از بررسی کارهای انجام شده در مورد موضوع پژوهش و یادداشت برداری از مطالب زمینه ای لازم برای تهیه گزارش نهائی طرح، پژوهشگر آماده است تا به بخش اجرای فعالیت های اصلی پژوهش بپردازد. این مرحله، در واقع، باید بر اساس طرح پژوهش تهیه شده و زمان بندی های تعیین شده اجرائی شود
		
		در حین انجام پروژه، باید از امکانات اینترنت حداکثر بهره برداری را نمود مانند:
		\begin{itemize}\rtlist
			\item لبست های پستی
			\item گروه های بحث و خبر (یوزنت)
			\item رایانامه
			\item نرم افزار، تصویر، ویدیو و صدا
			\item شبکه های اجتماعی
		\end{itemize}
	\end{frame}
	
	\begin{frame}{\frametitlefontsize  انجام بخش عملی پژوهش - لبست های پستی }
		\framefontsizelarge
		لبست های پستی برای بحث پیرامون یک موضوع توسط اعضای لیست به کار گرفته می شود و پژوهشگران عضو لیست مربوط به پژوهش خود با این کار در جریان بحث ها و اخبار جدید مربوط به موضوع پژوهش خود قرار می گیرند.\defaultvspace\\
		هر لیست دو نشانی مختلف دارد: 
		\begin{itemize}\rtlist
			\item نشانی اشتراک: دستورات باید به این فرستاده شوند
			\item نشانی لیست: پیام ها برای اعضا به این نشانی فرستاده می شوند
		\end{itemize}
		\persian
		برای اشتراک در یک لیست پستی باید به نشانی اشتراک درخواست فرستاد
	\end{frame}
	
	\begin{frame}{\frametitlefontsize  انجام بخش عملی پژوهش - مثال لیست های پستی }
		\framefontsizelarge
		\latin
		to: listserv@uga.cc.uga.edu\\
		subject: subscribe list-name\\
		subscribe list-name your family name your name\\
	\end{frame}
	
	\begin{frame}{\frametitlefontsize  انجام بخش عملی پژوهش - گروه های بحث و خبر (یوزنت) }
		\framefontsizelarge
		یوزنت یک سیستم جهانی شامل هزاران گروه بحث و خبر است که هریک برای اعلام نظر، پخش خبر و بحث پیرامون یک موضوع خاص تشکیل شده است. \defaultvspace\\
		مثلا وقتی فردی سوالی داشته باشد می تواند سوال خود را در گروه مربوطه بپرسد و افراد مختلف سرتاسر جهان به آن پاسخ دهند البته لازم است ابتدا به لبست سوالات رایج گروه مراجعه کند\\
		مثال: نام یک گروه خبری مربوط به لطیفه rec.humor
	\end{frame}

	\begin{frame}{\frametitlefontsize  انجام بخش عملی پژوهش - رایانامه }
		\framefontsizelarge
		برای درخواست کمک و یا گرفتن اطلاعات یا مطالب ویژه از افراد مشخص امروزه عموما از رایانامه استفاده می شود\\
		نامه ارسالی به یک فرد ناآشنا در حقیقت معرف فرستنده به او است و یک نامه بهتر تصویر بهتری از فرستنده در ذهن گیرنده نامه ایجاد می کند\defaultvspace\\
		نکات توصیه شده در نوشتن نامه:
		\begin{itemize}\rtlist
			\item هر نامه با نام گیرنده شروع و با نام فرستنده پایان یابد
			\item مطالب مهم تر پیام در چند سطر اول و مطلب اصلی ترجیحا در جمله اول آن قرار داده شود
			\item متن نامه به پاراگراف های کوتاه که با یک سطر فاصله سفید از هم جدا شده اند تقسیم شود
			\item از به کار بردن عنوان با نام خود خودداری شود
			\item پیام مودبانه باشد، به ویژه وقتی که از گیرنده درخواستی می شود
		\end{itemize}
	\end{frame}
	
	\begin{frame}{\frametitlefontsize  انجام بخش عملی پژوهش - رایانامه - ادامه }
		\framefontsizelarge
		\begin{itemize}\rtlist
			\item متن نامه با حروف بزرگ نوشته نشود
			\item برای تاکید روی کلمات یا عبارات، در دو طرف آن ** یا <> یا -- قرار داده شود
			\item در پاسخ به نامه ها بهتر است بخش های کوچکی از پیام اولیه که جوابش داده می شود آورده شود و در زیر آن پاسخ بیاید
			\item برای تورفتگی اول سطر از فاصله استفاده نشود، بلکه از TAB استفاده شود
			\item قسمت موضوع رایانامه باید حاوی عباراتی موجز، اما گویا و دقیق باشد. عبارات مبهم نباید در قسمت موضوع نوشته شود.
		\end{itemize}
	\end{frame}
	
	\begin{frame}{\frametitlefontsize  انجام بخش عملی پژوهش - نرم افزار، تصویر، ویدیو و صدا }
		\framefontsizelarge
		می توان برای پیشبرد پژوهش از این ابزارها و داده ها استفاده کرد. 
		
		امروزه برای انجام آزمایش و ارزیابی روش های مختلف حل مساله و مقایسه با کار دیگران بانک های اطلاعاتی مختلف ایجاد شده است که حاوی داده های عددی، تصویر، فیلم و یا صدا هستند
	\end{frame}
	
	\begin{frame}{\frametitlefontsize  انجام بخش عملی پژوهش - شبکه های اجتماعی }
		\framefontsizelarge
		در شبکه های اجتماعی مانند فیس بوک نیز می توان سوال را به صورت عمومی مطرح کرد و با درخواست کمک از افراد پاسخ دریافت نمود
	\end{frame}

	\begin{frame}{\frametitlefontsize  انجام بخش عملی پژوهش - اجرای پژوهش }
		\framefontsizelarge
		در این مرحله پژوهشگر آماده پژوهش است، باید براساس ماهیت موضوع پژوهش روشی انتخاب و آن را اجرا کند
		
		نکات: 
		\begin{itemize}\rtlist
			\item در اجرای همه بخش های پژوهش عملی باید دقیق بود
			\item در بخش نظری، همه کار چند بار و در چند زمان مختلف بررسی مجدد و کنترل شود
			\item پیاده سازی ها مجددا کنترل شوند
			\item در بخش ارزیابی کار انجام شده، بخصوص در طراحی و اجرای آزمایش های لازم توجه و دقت کافی شود و تلاش شود ارزیابی ها جامع تر و گسترده تر باشند
			\item به استاندارد بودن آزمایش ها و داده های مورد استفاده باید توجه ویژه نمود
			\item باید بتوان قبل از ارائه نتایج نسبت به صحت نتایج مطمئن شد
		\end{itemize}
	\end{frame}
	
	\section{\sectionfontsize ایجاد پیش نویس اولیه}	
	
	\begin{frame}{\frametitlefontsize  ایجاد پیش نویس اولیه - انشا نگارش اول }
		\framefontsizelarge
		نگارش پیش نویس اولیه ارائه قدم مهمی است که با انجام آن ترکیب اصلی مطالب و شکل ارائه مشخص می شود
		
		تا این مرحله، ساختار اولیه، یادداشت های مربوط به هر بخش ساختار، و نتایج پیاده سازی ها و کارهای عملی پژوهش همگی در دست هستند
		\begin{itemize}\rtlist
			\item تعیین لیست محتویات هر بخش
			\item نوشتن نسخه اول
		\end{itemize}
	\end{frame}
	
	\begin{frame}{\frametitlefontsize  ایجاد پیش نویس اولیه - انشا نگارش اول }
		\framefontsizelarge
		\textbf{تعیین لیست محتویات هر بخش}
		
		با توجه به مطالب موجود برای هر بخش، ابتدا باید لیستی  منطقی تهیه شود که در آن مطالبی که در هر بخش باید نوشته شود به ترتیب ذکر شوند
		
		 \textbf{نوشتن نسخه اول}
		 
		 باید در زمان هایی که نویسنده سرحال است، نوشتن را انحام دهد
		 
		 با داشتن لیست محتویات بخش ها و یادداشت ها، باید اجازه داد مطالب از ذهن فوران کنند و با سرعت تمام روی کاغذ آورده شوند
		 
		 نباید نگران نکات املائی یا انشائی و یا فنی بود زیرا نگارش نوشته مطلوب و نهائی نیست
		 
		 نگارش باید به صورت تکاملی باشد (ابتدا چیزی نوشته شود و سپس اصلاح شود)
	\end{frame}
	
	\begin{frame}{\frametitlefontsize  ایجاد پیش نویس اولیه - انشا نگارش اول }
		\framefontsizelarge
		نباید در مورد یک مطلب زیاده گوئی کرد چون باعث پاره شدن رشته فکری می شود و می تواند به جریان ایده ها و به پیوستگی نوشته لطمات اساسی وارد کند
		
		در مرحله بازخوانی اگر احساس شود در جائی باید توضیحات بیشتری داده شود به راحتی آن را انجام می دهیم
		
		نباید وقتی مطلبی نوشته شد، نویسنده فورا برگشته و آنرا بخواند زیرا به احتمال زیاد تصور می کند که مطلب باید اصلاح شود. باید همه بخش ها را نوشت و پس از پایان نگارش اولیه، کار بازخوانی را انجام داد
		
		بهتر است برای رعایت پیشنیاز و وابستگی بخش ها نگارش از اول ساختار شروع شود 
		
		لزومی ندارد که همه مطالب یادداشت برداری شده نوشته شود
	\end{frame}
	
	\begin{frame}{\frametitlefontsize  ایجاد پیش نویس اولیه - بازخوانی و اصلاح نخستین نگارش }
		\framefontsizelarge
		در مرحله بازخوانی، باید سبک، سازمان، دقیق بودن، صحیح بودن، و جامع بودن مورد بررسی قرار گیرد
		
		کلمات مبهم باید با کلمات دقیق و جملات و عبارات پیچیده باید با جملات روشن و واضح جایگزین شوند
		
		کار دیگر، تنظیم توضیحات متن است
		
		اضافه کردن توضیحات در جاهایی که توضیحات کافی ارائه نشده است و یا حذف توضیحات اضافی
		
		مطلب دیگر هماهنگی سبک نگارش بخش های مختلف است (چون به احتمال زیاد همه در یک نشست نوشته نشده اند)
		
		پس از اصلاح نگارش اول مطلب، یک پیش نویس اولیه که از نظر محتویات قابل قبول است در دست خواهد بود
	\end{frame}

\end{persian}
\end{document}