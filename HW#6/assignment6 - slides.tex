\documentclass[14pt]{beamer}

\mode<presentation>
\usetheme[progressbar=frametitle]{metropolis}

\usepackage{booktabs}
\usepackage[scale=2]{ccicons}
\usepackage{xspace}
\usepackage{xepersian}

\defbeamertemplate*{title page}{customized}[1][]
{	
	\usebeamercolor[fg]{titlegraphic}\inserttitlegraphic\par
	\raggedleft\usebeamerfont{title}\inserttitle\par
	\usebeamerfont{subtitle}\usebeamercolor[fg]{subtitle}\insertsubtitle\par
	\bigskip
	\usebeamerfont{author}\insertauthor\par
	\usebeamerfont{institute}\insertinstitute\par
	\usebeamerfont{date}\insertdate\par
}	

\title{یادداشت برداری و ابزار آن، انجام بخش عملی پژوهش و نوشتن نسخه اولیه گزارش}
\subtitle{\color{brown} روش تحقیق و گزارش نویسی}
\date{17 فروردین 1395}
\author{سیاوش کاوسی}
\institute{دانشگاه صنعتی امیرکبیر}
\titlegraphic{\hfill\includegraphics[height=1.5cm]{logo}}

\makeatletter
\newcommand{\rtlist}{\raggedleft\rightskip\@totalleftmargin} 
\makeatother
\newcommand{\sectionfontsize}{\fontsize{22pt}{0pt}\selectfont}
\newcommand{\framefontsizelarge}{\fontsize{18pt}{0pt}\selectfont}
\newcommand{\frametitlefontsize}{\fontsize{20pt}{0pt}\selectfont}
\newcommand{\defaultvspace}{\vspace{5mm}}

% xepersian font settings 
\settextfont{Adobe Arabic}\fontsize{12pt}{0pt}


\begin{document}
\begin{persian}
	\maketitle
	\everypar{\rightskip\rightmargin}		
	
	\section{\sectionfontsize یادداشت برداری و ابزار آن}	
	
	\begin{frame}{\frametitlefontsize یادداشت برداری}
		\framefontsizelarge
		تسهیل یادداشت برداری با کمک ساختار تهیه شده در زمان مطالعه منابع و نوشتن روی آنها \defaultvspace\\
		انجام یادداشت برداری بر روی \textbf{فیش}\defaultvspace\\
		فیش ها می توانند هر اندازه ای باشند  \textbf{اما} بهتر است اندازه  $15*10$ یا  $21*15$ سانتیمتر انتخاب شود که اولی برای یادداشت های کوتاه تر و دومی برای یادداشت های بلند تر
	\end{frame}	
	
	\begin{frame}{\frametitlefontsize یادداشت برداری - ادامه}
		\framefontsizelarge
		عکس مربوطه!!!
	\end{frame}	
	
\end{persian}
\end{document}