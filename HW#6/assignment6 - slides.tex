\documentclass[14pt]{beamer}

\mode<presentation>
\usetheme[progressbar=frametitle]{metropolis}

\usepackage{booktabs}
\usepackage[scale=2]{ccicons}
\usepackage{xspace}
\usepackage{xepersian}

\defbeamertemplate*{title page}{customized}[1][]
{	
	\usebeamercolor[fg]{titlegraphic}\inserttitlegraphic\par
	\raggedleft\usebeamerfont{title}\inserttitle\par
	\usebeamerfont{subtitle}\usebeamercolor[fg]{subtitle}\insertsubtitle\par
	\bigskip
	\usebeamerfont{author}\insertauthor\par
	\usebeamerfont{institute}\insertinstitute\par
	\usebeamerfont{date}\insertdate\par
}	

\title{یادداشت برداری و ابزار آن، انجام بخش عملی پژوهش و نوشتن نسخه اولیه گزارش}
\subtitle{\color{brown} روش تحقیق و گزارش نویسی}
\date{17 فروردین 1395}
\author{سیاوش کاوسی}
\institute{دانشگاه صنعتی امیرکبیر}
\titlegraphic{\hfill\includegraphics[height=1.5cm]{logo}}

\makeatletter
\newcommand{\rtlist}{\raggedleft\rightskip\@totalleftmargin} 
\makeatother
\newcommand{\sectionfontsize}{\fontsize{22pt}{0pt}\selectfont}
\newcommand{\framefontsizelarge}{\fontsize{18pt}{0pt}\selectfont}
\newcommand{\frametitlefontsize}{\fontsize{20pt}{0pt}\selectfont}
\newcommand{\defaultvspace}{\vspace{5mm}}

% xepersian font settings 
\settextfont{Adobe Arabic}\fontsize{12pt}{0pt}


\begin{document}
\begin{persian}
	\maketitle
	\everypar{\rightskip\rightmargin}		
	
	\section{\sectionfontsize یادداشت برداری و ابزار آن}	
	
	\begin{frame}{\frametitlefontsize یادداشت برداری}
		\framefontsizelarge
		تسهیل یادداشت برداری با کمک ساختار تهیه شده در زمان مطالعه منابع و نوشتن روی آنها \defaultvspace\\
		انجام یادداشت برداری بر روی \textbf{فیش}\defaultvspace\\
		فیش ها می توانند هر اندازه ای باشند  \textbf{اما} بهتر است اندازه  $15*10$ یا  $21*15$ سانتیمتر انتخاب شود که اولی برای یادداشت های کوتاه تر و دومی برای یادداشت های بلند تر
	\end{frame}	
	
	\begin{frame}{\frametitlefontsize یادداشت برداری - مثالی از فیش}
		\framefontsizelarge
		عکس مربوطه!!!
	\end{frame}	
	
	\begin{frame}{\frametitlefontsize یادداشت برداری - ساختار فیش}
		\framefontsizelarge
		
		\begin{itemize}\rtlist
			\item در سمت چپ بالا، ذکر نوع یادداشت  
			\defaultvspace\\
			\item در پایین، ذکر نکاتی که پژوهشگر مایل است در مورد مطلب روی فیش فراموش نشود. توصیه می شود مطالبی که ناگهان به ذهن می رسد و چند لحظه بعد فراموش می شود در این مکان نوشته شود
		\end{itemize}
	\end{frame}	
	
	\begin{frame}{\frametitlefontsize یادداشت برداری - نکات مهم در مورد فیش}
		\framefontsizelarge
		روی هر فیش تنها یک مطلب باید نوشته شود نه بیشتر\\
	   نوشتن بیش از یک مطلب روی یک فیش مرحله بعد، دسته بندی یادداشت ها را با دشواری روبرو خواهد ساخت\defaultvspace\\
	   
	   اگر مطلب تا آخر یک فیش به پایان نرسد، در انتهای این فیش و ابتدای فیش بعدی از یک علامت پیکان افقی 
	   ($\leftarrow$)
	   استفاده می شود تا ادامه دار بودن مطلب مشخص گردد	
	\end{frame}
	
	\begin{frame}{\frametitlefontsize یادداشت برداری - انواع}
		\framefontsizelarge
		\begin{itemize}\rtlist
			\item نقل مستقیم
			\item نقل غیر مستقیم
			\item ترجمه با یا بدون خلاصه کردن
			\item یادداشت برداری
		\end{itemize}
	\end{frame}
	
	\begin{frame}{\frametitlefontsize یادداشت برداری - نقل مستقیم}
		\framefontsizelarge
		یادداشت برداری عین متن با همه تاکیدات آن\defaultvspace\\
		در زمان یادداشت برداری از متون دقیق مانند دستورالعمل ها، قوانین و امثال آن لازم می شود، یا وقتی مطلب بسیار خوب بیان شده و هرگونه تغییر در آن، آنرا خدشه دار می کند\defaultvspace\\
		خلاصه کردن نقل مستقیم با حذف قسمت هایی از آن و گذاشتن سه نقطه به جای آن انجام می شود
	\end{frame}
	
	\begin{frame}{\frametitlefontsize یادداشت برداری - نقل غیر مستقیم}
		\framefontsizelarge
	   عین کلمات و عبارات متن اصلی اهمیتی برای پژوهشگر ندارد لذا برای هماهنگ ساختن نوشته ها پژوهشگر ترجیح می دهد مطلب را با بیان خود بازگو کند\defaultvspace\\
		پژوهشگر باید دقت داشته باشد تا مفاهیم متن در بازگوئی تغییر داده نشود\defaultvspace\\
		نقل غیرمستقیم همراه با تلخیص، پژوهشگر باید از ذکر مثال ها و شرح صرف نظر نمود و ایده های اصلی متن را با زبان خود نوشت
	\end{frame}
	
	\begin{frame}{\frametitlefontsize یادداشت برداری - نکات مهم}
		\framefontsizelarge
		یادداشت برداری باید در حدی کامل باشد که پژوهشگر را از مراجعه مجدد به مرجع بی نیاز کند\\
		این قدم گرچه زمان بر است اما در انجام مطلوب تر کار بسیار موثر است و با تسریعی که در قدم های بعدی ایجاد می کند مجموعا اتلاف وقت ببار نمی آورد\defaultvspace\\
		توصیه می شود که مطالب را با خط کشیدن انتخاب نکنیم اما اگر مطلب طولانی باشد می توان کپی مطلب را تهیه نمود و ضمیمه فیش های مربوطه کرد
	\end{frame}
	
	\begin{frame}{\frametitlefontsize یادداشت برداری - راهکارهای الکترونیکی}
		\framefontsizelarge
		اگر مطالعه با راهکارهای الکترونیکی انجام شده باشد، طبیعی است که یادداشت برداری نیز با همین رویکرد انجام پذیرد. \defaultvspace\\
		می توان از نرم افزارهای واژه پرداز یا نرم افزارهای یادداشت برداری برای تسریع کار استفاده کرد (مثلا نرم افزار وان نوت)
	\end{frame}
	
	\begin{frame}{\frametitlefontsize  یادداشت برداری - دسته بندی و ارزیابی }
		\framefontsizelarge
		دسته بندی فیش ها بر حسب بخش ها پس از مطالعه و یادداشت برداری از همه منابع
		
		بررسی و تصمیم گیری در مورد تناسب مجموعه اطلاعات گردآوری شده برای هر بخش و اهمیت آن بخش

		یادداشت برداری از منابع جدید در جهت کاهش نقصان بخش هایی که مطالب کافی بدست نیامده
		
		اگر منابع مناسب پیدا نشود، تنها راه حل تغییر طرح و ساختار ارائه و حذف مباحث مورد بحث است
		
		ارزیابی ساختار ارائه شده و مطالب تهیه شده برای هر بخش و \textbf{تغییر} در صورت لزوم
	\end{frame}
	
\end{persian}
\end{document}