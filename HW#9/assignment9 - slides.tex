\documentclass[14pt]{beamer}

\mode<presentation>
\usetheme[progressbar=frametitle]{metropolis}

\usepackage{booktabs}
\usepackage[scale=2]{ccicons}
\usepackage{xspace}
\usepackage{xepersian}

\defbeamertemplate*{title page}{customized}[1][]
{	
	\usebeamercolor[fg]{titlegraphic}\inserttitlegraphic\par
	\raggedleft\usebeamerfont{title}\inserttitle\par
	\usebeamerfont{subtitle}\usebeamercolor[fg]{subtitle}\insertsubtitle\par
	\bigskip
	\usebeamerfont{author}\insertauthor\par
	\usebeamerfont{institute}\insertinstitute\par
	\usebeamerfont{date}\insertdate\par
}	

\title{نکات ویژه در انواع ارائه نوشتاری - سازماندهی و زمانبندی ارائه گفتاری}
\subtitle{\color{brown} روش تحقیق و گزارش نویسی}
\date{17 فروردین 1395}
\author{سیاوش کاوسی}
\institute{دانشگاه صنعتی امیرکبیر}
\titlegraphic{\hfill\includegraphics[height=1.5cm]{logo}}

\makeatletter
\newcommand{\rtlist}{\raggedleft\rightskip\@totalleftmargin} 
\makeatother
\newcommand{\sectionfontsize}{\fontsize{22pt}{0pt}\selectfont}
\newcommand{\framefontsizelarge}{\fontsize{18pt}{0pt}\selectfont}
\newcommand{\frametitlefontsize}{\fontsize{20pt}{0pt}\selectfont}
\newcommand{\defaultvspace}{\vspace{5mm}}

% xepersian font settings 
\settextfont{Adobe Arabic}\fontsize{12pt}{0pt}

\begin{document}
\begin{persian}
	\maketitle
	\everypar{\rightskip\rightmargin}		
	
	\section{\sectionfontsize نکات ویژه در انواع ارائه نوشتاری}	
	
	\begin{frame}[plain]{\frametitlefontsize نکات ویژه در انواع ارائه نوشتاری}
		\framefontsizelarge
		مطالب مطرح شده در دو درس گذشته مطالب کلی در تهیه گزارش نوشتاری هستند
		
		ارائه های نوشتاری گرچه دارای جنبه های یکسانی هستند اما در مورد هر یک نکات ویژه ای نیز مطرح است که باید به آن ها توجه کرد
		
		انواع ارائه های نوشتاری رایج
		\begin{itemize}\rtlist
			\item سمینار دانشجوئی
			\item مقاله برای نشریات علمی
			\item مقاله برای همایش ها
			\item پایان نامه های تحصیلی
			\item گزارش وضعیت یا پیشرفت طرح پژوهشی
		\end{itemize}
	\end{frame}	
	
	\begin{frame}[plain]{\frametitlefontsize نکات ویژه در انواع ارائه نوشتاری}
		\framefontsizelarge
		\begin{itemize}\rtlist
			\item گزارش پروژه بر اساس روش علمی
			\item گزارش مطالعه امکان سنجی
			\item گزارش ارزیابی و توصیه
			\item گزارش حادثه
			\item گزارش بازدید
			\item گزارش سفر
			\item گزارش یک آزمایش یا بررسی عملی
			\item گزارش کارآموزی
		\end{itemize}
		\persian
		نکات مهم مرتبط با بعضی ارائه های نوشتاری ضروری مانند دستورالعمل ها، مشخصات فنی، حرفه نامه، و نامه های اداری نیز مطرح می گردد
	\end{frame}	

	\section{\sectionfontsize مقاله برای همایش ها}	

	\begin{frame}[plain]{\frametitlefontsize مقاله برای همایش ها}
		\framefontsizelarge
		ساختار مشابه مقالات نشریات علمی دارند، با این تفاوت که مقالات همایش ها معمولا کوتاه تر هستند و جامعیت مقالات نشریات را ندارند
		
		حداکثر طول مقالات همایش ها معمولا چهار، شش، یا هشت صفحه است که باید اکیدا رعایت شود
		
		مقالات همایش ها قاعدتا باید حاوی ایده های جدید و نوآوری باشند
		
		مقالات ارسالی به همایش ها باید هم از نظر ساختاری و شکلی و هم از نظر محتوای مقاله به فرم نهائی بسیار نزدیک باشند زیرا عمدتا زمانی برای اصلاح و داوری مجدد مقاله نیست
	\end{frame}	
	
	\begin{frame}[plain]{\frametitlefontsize مقاله برای همایش ها - بخش بندی و محتویات}
		\framefontsizelarge
		بخش بندی و محتویات یک مقاله همایش نمونه
		\begin{itemize}\rtlist
			\item چکیده
			\item مقدمه
			\item کارهای مرتبط
			\item روش پیشنهادی
			\item ارزیابی
			\item نتیجه گیری و چشم انداز آینده
		\end{itemize}
	\end{frame}	
	
	\begin{frame}[plain]{\frametitlefontsize مقاله برای همایش ها - چکیده}
		\framefontsizelarge
		بسته به طول مقاله 75 الی 300 کلمه است
		
		بیان کننده مساله مورد بحث مقاله، راه حل پیشنهادی، دلایل موثربودن راه حل، ارزیابی کلی روش پیشنهادی است 
		
		در یک پاراگراف و با فعل های گذشته نوشته می شود
		
		نباید حاوی ارجاع به مستندات دیگر، مثال، تصویر، و یا جدول باشد
	\end{frame}	
	
	\begin{frame}[plain]{\frametitlefontsize مقاله برای همایش ها - مقدمه}
		\framefontsizelarge
		شرح مساله موردنظر در مقاله در 1 الی 3 پاراگراف
		
		ذکر دلایل حل نشدن مساله تا این زمان و موثر نبودن راه حل های قبلی
		
		توضیح ارزش راه حل پیشنهادی و دلایل بهتر بودن آن از راه حل های موجود
		
		شرح مختصر بخش های بعدی مقاله
	\end{frame}	

	\begin{frame}[plain]{\frametitlefontsize مقاله برای همایش ها - کارهای مرتبط}
		\framefontsizelarge
		تلاش های دیگری که در جهت حل مساله به ویژه در چند ساله اخیر صورت گرفته و دلایل برتری روش پیشنهادی بر آن کارها به صورت مختصر بیان می گردد
		\defaultvspace
		اگر تاکنون راه حلی برای مساله موردنظر ارائه نشده است، تلاش هایی که برای حل مسائل مرتبط یا مسائل بسیار شبیه به مساله موردنظر صورت گرفته پذیرفته و ارتباط یا شباهت آنها با این کار می تواند مطرح شود
	\end{frame}	
	
	\begin{frame}[plain]{\frametitlefontsize مقاله برای همایش ها - روش ارزیابی}
		\framefontsizelarge
		راه حل های پیشنهادی برای حل مساله ارائه می شود 
		
		در صورت نیاز، این قسمت می تواند شامل زیربخش هایی باشد
		
		ارائه جزئیات زیر و دقیق و یا اثبات های ریاضی در این بخش ضروری نیست
	\end{frame}	

	\begin{frame}[plain]{\frametitlefontsize مقاله برای همایش ها - ارزیابی}
		\framefontsizelarge
		روش ارزیابی راه حل، معیار های سنجش کارایی، پارامترهای سنجش کارایی، و طراحی آزمایش های لازم مطرح می گردد
		
		عملکرد روش و مقایسه عملکرد آن با عملکرد دیگر روش ها ارائه می شود
		
		نتایج ارائه و تفسیر می شوند
		
		میزان برتری راه حل و چگونگی و دلایل برتری مطرح می شوند
		
		ویژگی های برجسته روش و شرایط و محدودیت های آن مورد تاکید قرار می گیرند، و توضیح های موردنیاز ارائه می شوند
	\end{frame}	
	
	\begin{frame}[plain]{\frametitlefontsize مقاله برای همایش ها - نتیجه گیری و چشم انداز آینده}
		\framefontsizelarge
		در این بخش ابتدا باید مساله حل شده ترجیحا در یک و حداکثر در سه جمله بیان شود
		
		پس از آن باید راه حل مساله در یک یا دو جمله توضیح داده شود و دلایل ارزشمند بودن آن ذکر گردد
		
		در انتها نیز کارهایی که می توان در ادامه این کار انجام داد - مانند اصلاح و بهبود راه حل، بکار گرفتن راه حل در موارد مشکل تر و واقعی تر مساله مطرح شده - را مطرح می نماید
		
		نکات مهم برای مقالات ارسالی به هر کنفرانس در راهنمای تهیه مقالات کنفرانس ذکر می شود
	\end{frame}	
	
	\begin{frame}[plain]{\frametitlefontsize مقاله برای همایش ها - نتیجه گیری و چشم انداز آینده}
		\framefontsizelarge
		ارزیابی مقالات برای پذیرش در همایش ها بر مبنای مقاله کامل و در بعضی بر مبنای یک چکیده مبسوط یا خلاصه انجام می پذیرد
		
		اگر پذیرش بر مبنای خلاصه مقاله باشد، نویسندگان باید تلاش نمایند که خلاصه علاقه خواننده را جلب کند و کسالت آور نباشد
		
		انتخاب یک عنوان کوتاه اما جاذب و چشمگیر پیشنهاد می شود
		
		خلاصه مقاله باید با بیان صورت مساله و سوال پژوهش آغاز شود و با اهمیت و انگیزه انجام پژوهش ادامه داده شود
	\end{frame}	
	
	\begin{frame}[plain]{\frametitlefontsize مقاله برای همایش ها - نتیجه گیری و چشم انداز آینده}
		\framefontsizelarge
		در خلاصه باید به جای بیان طولانی موضوع کلی پژوهش بیشتر به مشخصات ویژه کار مطرح در مقاله پرداخت
		
		پس از آن باید روش انجام پژوهش ارائه گردد و نتایج کلی حاصل از تحقیق به طور واضح و دقیق بیان شوند و اهمیت این نتایج و چکونگی مفید بودن آن ها مشخص گردد
		
		طول خلاصه باید بین 300 الی 500 کلمه باشد
		
		در انتها باید بین پنج الی ده واژه کلیدی که منعکس کننده جنبه های اصلی کار هستند، به عنوان واژه های کلیدی به خلاصه مقاله اضافه گردد
	\end{frame}	
	
		\section{\sectionfontsize گزارش پایان نامه تحصیلی}	
	
	\begin{frame}[plain]{\frametitlefontsize مقاله برای همایش ها - گزارش پایان نامه تحصیلی}
		\framefontsizelarge
		گزارش پایان نامه تحصیلی در واقع گزارش انجام یک پروژه یا کار تحقیقاتی نسبتا طولانی است که باید به صورت مناسب ارائه شود
		
		محتویات و ساختار مطالب باید بر طبق اصولی مشخص انتخاب گردند
		
		دانشگاه های مختلف هر یک برای خود دفترچه راهنمائی برای نگارش پایان نامه تهیه دیده اند که نکات مختلف مورد نظر در آن گنجانده شده است
		
		هر فصل پایان نامه با یک پاراگراف بدون عنوان \textbf{مقدمه فصل} شروع می شود که در آن مطالب فصل و اجزا آن شرح داده می شوند
	\end{frame}	
	
	\begin{frame}[plain]{\frametitlefontsize مقاله برای همایش ها - محتویات و ساختار پایان نامه}
		\framefontsizelarge
		\textbf{فصل اول - مقدمه}
	\end{frame}	

\end{persian}
\end{document}