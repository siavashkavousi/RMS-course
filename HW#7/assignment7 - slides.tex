\documentclass[14pt]{beamer}

\mode<presentation>
\usetheme[progressbar=frametitle]{metropolis}

\usepackage{booktabs}
\usepackage[scale=2]{ccicons}
\usepackage{xspace}
\usepackage{xepersian}

\defbeamertemplate*{title page}{customized}[1][]
{	
	\usebeamercolor[fg]{titlegraphic}\inserttitlegraphic\par
	\raggedleft\usebeamerfont{title}\inserttitle\par
	\usebeamerfont{subtitle}\usebeamercolor[fg]{subtitle}\insertsubtitle\par
	\bigskip
	\usebeamerfont{author}\insertauthor\par
	\usebeamerfont{institute}\insertinstitute\par
	\usebeamerfont{date}\insertdate\par
}	

\title{عناصر گزارش نوشتاری نهائی}
\subtitle{\color{brown} روش تحقیق و گزارش نویسی}
\date{17 فروردین 1395}
\author{سیاوش کاوسی}
\institute{دانشگاه صنعتی امیرکبیر}
\titlegraphic{\hfill\includegraphics[height=1.5cm]{logo}}

\makeatletter
\newcommand{\rtlist}{\raggedleft\rightskip\@totalleftmargin} 
\makeatother
\newcommand{\sectionfontsize}{\fontsize{22pt}{0pt}\selectfont}
\newcommand{\framefontsizelarge}{\fontsize{18pt}{0pt}\selectfont}
\newcommand{\frametitlefontsize}{\fontsize{20pt}{0pt}\selectfont}
\newcommand{\defaultvspace}{\vspace{5mm}}

% xepersian font settings 
\settextfont{Adobe Arabic}\fontsize{12pt}{0pt}


\begin{document}
\begin{persian}
	\maketitle
	\everypar{\rightskip\rightmargin}		
	
	\section{\sectionfontsize اجزاء معمول در گزارش های مهندسی}	
	
	\begin{frame}[allowframebreaks]{\frametitlefontsize اجزاء معمول در گزارش های مهندسی}
		\framefontsizelarge
		در انواع گزارش های مهندسی باید اجزائی در ابتدا و عناصری در انتهای گزارش اضافه شوند و در بدنه اصلی گزارش نیز ضوابط خاصی رعایت شوند
		
		\begin{itemize}\rtlist
			\item جلد و شیرازه
			\item صفحه عنوان
			\item صفحات اهدا و تشکر
			\item چکیده
			\item فهرست مطالب
			\item فهرست اشکال
			\item فهرست جداول
			\item بدنه اصلی گزارش
			\item لبست مراجع
			\item پیوست ها
		\end{itemize}
	\end{frame}	
	
	\begin{frame}[plain]{\frametitlefontsize  اجزاء معمول در گزارش های مهندسی - جلد و شیرازه}
		\framefontsizelarge
		وجود یک جلد خاوی اطلاعات مهم گزارش، گزارش را مورد توجه بیشتر و مقبولیت بالاتر قرار می دهد
		
		در گزارش هایی مانند پایان نامه های دانشگاهی مرسوم است که از جلدهای محکم مقوائی و صحافی استفاده شود
		
		اطلاعات روی جلد:
		\begin{itemize}\rtlist
			\item عنوان گزارش
			\item موسسه ای که گزارش برای آن تهیه شده
			\item نام کامل ارائه دهنده
			\item موسسه محل اشتغال ارائه دهنده
			\item شماره یا کد گزارش
			\item تاریخ ارائه گزارش
		\end{itemize}
	\end{frame}	
	
	\begin{frame}[plain]{\frametitlefontsize  اجزاء معمول در گزارش های مهندسی - صفحه عنوان}
		\framefontsizelarge
		مرسوم است که پس از جلد یک صفحه سفید قرار داده شود و پس از آن در بعضی گزارشها اطلاهات روی جلد عینا در یک \textbf{صفحه عنوان} تکرار می گردد
		
		وجود صفجه عنوان در پایان نامه های دانشگاهی الزامی است
	\end{frame}	
	
\end{persian}
\end{document}