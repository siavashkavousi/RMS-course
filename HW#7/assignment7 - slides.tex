\documentclass[14pt]{beamer}

\mode<presentation>
\usetheme[progressbar=frametitle]{metropolis}

\usepackage{booktabs}
\usepackage[scale=2]{ccicons}
\usepackage{xspace}
\usepackage{xepersian}

\defbeamertemplate*{title page}{customized}[1][]
{	
	\usebeamercolor[fg]{titlegraphic}\inserttitlegraphic\par
	\raggedleft\usebeamerfont{title}\inserttitle\par
	\usebeamerfont{subtitle}\usebeamercolor[fg]{subtitle}\insertsubtitle\par
	\bigskip
	\usebeamerfont{author}\insertauthor\par
	\usebeamerfont{institute}\insertinstitute\par
	\usebeamerfont{date}\insertdate\par
}	

\title{عناصر گزارش نوشتاری نهائی}
\subtitle{\color{brown} روش تحقیق و گزارش نویسی}
\date{17 فروردین 1395}
\author{سیاوش کاوسی}
\institute{دانشگاه صنعتی امیرکبیر}
\titlegraphic{\hfill\includegraphics[height=1.5cm]{logo}}

\makeatletter
\newcommand{\rtlist}{\raggedleft\rightskip\@totalleftmargin} 
\makeatother
\newcommand{\sectionfontsize}{\fontsize{22pt}{0pt}\selectfont}
\newcommand{\framefontsizelarge}{\fontsize{18pt}{0pt}\selectfont}
\newcommand{\frametitlefontsize}{\fontsize{20pt}{0pt}\selectfont}
\newcommand{\defaultvspace}{\vspace{5mm}}

% xepersian font settings 
\settextfont{Adobe Arabic}\fontsize{12pt}{0pt}

\begin{document}
\begin{persian}
	\maketitle
	\everypar{\rightskip\rightmargin}		
	
	\section{\sectionfontsize اجزاء معمول در گزارش های مهندسی}	
	
	\begin{frame}[allowframebreaks]{\frametitlefontsize اجزاء معمول در گزارش های مهندسی}
		\framefontsizelarge
		در انواع گزارش های مهندسی باید اجزائی در ابتدا و عناصری در انتهای گزارش اضافه شوند و در بدنه اصلی گزارش نیز ضوابط خاصی رعایت شوند
		
		\begin{itemize}\rtlist
			\item جلد و شیرازه
			\item صفحه عنوان
			\item صفحات اهدا و تشکر
			\item چکیده
			\item فهرست مطالب
			\item فهرست اشکال
			\item فهرست جداول
			\item بدنه اصلی گزارش
			\item لبست مراجع
			\item پیوست ها
		\end{itemize}
	\end{frame}	
	
	\begin{frame}[plain]{\frametitlefontsize  اجزاء معمول در گزارش های مهندسی - جلد و شیرازه}
		\framefontsizelarge
		وجود یک جلد خاوی اطلاعات مهم گزارش، گزارش را مورد توجه بیشتر و مقبولیت بالاتر قرار می دهد
		
		در گزارش هایی مانند پایان نامه های دانشگاهی مرسوم است که از جلدهای محکم مقوائی و صحافی استفاده شود
		
		اطلاعات روی جلد:
		\begin{itemize}\rtlist
			\item عنوان گزارش
			\item موسسه ای که گزارش برای آن تهیه شده
			\item نام کامل ارائه دهنده
			\item موسسه محل اشتغال ارائه دهنده
			\item شماره یا کد گزارش
			\item تاریخ ارائه گزارش
		\end{itemize}
	\end{frame}	
	
	\begin{frame}[plain]{\frametitlefontsize  اجزاء معمول در گزارش های مهندسی - صفحه عنوان}
		\framefontsizelarge
		مرسوم است که پس از جلد یک صفحه سفید قرار داده شود و پس از آن در بعضی گزارشها اطلاهات روی جلد عینا در یک \textbf{صفحه عنوان} تکرار می گردد
		
		وجود صفجه عنوان در پایان نامه های دانشگاهی الزامی است
	\end{frame}	
	
	\begin{frame}[plain]{\frametitlefontsize  اجزاء معمول در گزارش های مهندسی - صفحات اهدا و تشکر}
		\framefontsizelarge
		در بعضی گزارش ها مانند پایان نامه های دانشگاهی، می توان صفحاتی جهت اهدا گزارش به بعضی افراد و یا تشکر و قدردانی از دیگران قرار داد که به صورت پی در پی بعد از عنوان قرار می گیرند
	\end{frame}	
	
	\begin{frame}[plain]{\frametitlefontsize  اجزاء معمول در گزارش های مهندسی - چکیده}
		\framefontsizelarge
		خلاصه محتویات گزارش را مطرح می کند و در حقیقت انتظارات خواننده را تنظیم می نماید
		چکیده شامل:
		\begin{itemize}\rtlist
			\item بیان موضوع و هدف گزارش
			\item ذکر محدوده مطرح در کار
			\item روش حل مساله
			\item روش ارزیابی
			\item نتایج حاصل
		\end{itemize}
		
		در مقالات علمی ممکن است بین 50 تا 300 کلمه و در گزارش های مهندسی طولانی تر باشد
		
		در حد امکان باید از بکار بردن علائم اختصاری، کوته نوشت ها، شکل، جدول و فرمول پرهیز نمود و به مراجع ارجاع نداد
	\end{frame}	
	
	\begin{frame}[plain]{\frametitlefontsize  اجزاء معمول در گزارش های مهندسی - فهرست مطالب}
		\framefontsizelarge
		باید به فضابندی، به داخل بردن زیرعناوین نسبت به عناوین اصلی تر، بزرگتر گرفتن اندازه قلم در عناوین اصلی تر، و همسان بودن و هم ستون بودن عناوین هم سطح توجه نمود
		
		شماره صفحات باید به گونه ای زیر هم باشند که آخرین رقم سمت چپ آنها در یک ستون قرار گیرند
		
		برای افزایش خوانائی و ایجاد توزیع بهتر عناوین در امتداد عمودی می توان در بالای عناوین اصلی یک سطر فاصله قرار داد
		
		شماره، عناوین، و شماره صفحات هر بخش یا زیربخش عینا مانند آنچه در متن آمده در فهرست مطالب وارد گردد
	\end{frame}	
	
	\begin{frame}[plain]{\frametitlefontsize  اجزاء معمول در گزارش های مهندسی - فهرست اشکال - جداول}
		\framefontsizelarge
		کلیه مطالبی که در فهرست مطالب بیان شد در مورد این فهرست نیز باید رعایت شود
		
		اگر گزارش چند عکس داشته باشد، این فهرست بطور مستقل تشکیل داده می شود. در غیر اینصورت یا این فهرست حذف می شود یا اگر چند جدول داشته باشیم، فهرست مشترک با عنوان \textbf{فهرست اشکال و جداول} تشکیل داده می شود
		\defaultvspace
		
		موارد بالا برای فهرست جداول نیز صادق است
	\end{frame}	
	
	\begin{frame}[allowframebreaks]{\frametitlefontsize  اجزاء معمول در گزارش های مهندسی - بدنه اصلی گزارش}
		\framefontsizelarge
		فصل اول گزارش به \textbf{مقدمه} تعلق دارد. پس از آن فصل های مختلف به ترتیب آورده می شوند و سرانجام فصل آخر با عنوان \textbf{نتیجه گیری و پیشنهادها} به بدنه اصلی گزارش خاتمه می دهد
		\defaultvspace
		
		نکات مربوط به بدنه اصلی گزارش
		\begin{itemize}\rtlist
			\item در شماره گذاری صفحات، همه صفحات باید شمرده شوند، گرچه شماره بعضی از آن ها روی صفحه نوشته نشود. در گزارش های طولانی می توان صفحات هر فصل را بطور مستقل و با ترکیب شماره فصل شماره زد
			\item باید سعی شود از شکل، نمودار، و جدول در حد امکان برای انتقال راحت تر اطلاعات در گزارش و تصویری کردن گزارش استفاده نمود
			\item شکل ها و نمودارها باید ساده، واضح، و بدون ابهام باشند. علائم، نماد ها، مقیاس ها و سایر اطلاعات لازم باید در روی آنها قرار داده شوند، بطوریکه خواننده بدون نیاز به رجوع به متن، آنها را درک کند
			
			هر شکل که در گزارش ظاهر می شود باید حتما بطور صریح در متن به آن ارجاع داده شده باشد. محل درج شکل در گزارش حتما باید پس از رجوع در متن و ترجیحا در اولین محل ممکن پس از آن باشد.
			
			بهتر است از جمع کردن شکل ها در انتهای گزارش اجتناب شود
			
			اندازه هر شکل باید متناسب با محتویات آن شکل باشد، شکل ها باید بگونه ای در صفحه قرار داده شوند که در حاشیه های چهار طرف نروند
			
			برای تولید راحت تر گزارش ها می توان شکل ها را در صفحات جداگانه قرار داد اما قرار دادن شکل بین متون گزارش شکل حرفه ای تری را به خود می گیرد
			
			اگر ابعاد شکل طوری باشد که لازم شود در طول صفحه قرار داده شود، برای صفحات سمت راست گزارش، بالای شکل و برای صفحات سمت چپ، پایین شکل به سمت شیرازه قرار می گیرند
			\item جداول اطلاعات را به صورت دسته بندی شده ارائه می دهند. در هر جا که ایجاد جدول میسر باشد، انجام این کار توصیه می شود. نکات مربوط به شکل ها برای جدول ها نیز صادق است
			
			تفاوت مطرح بین شکل و جدول آن است که شماره و عنوان جدول در بالای آن قرار می گیرد اما در شکل ها در زیر شکل قرار می گیرد
			\item وجود فضاهای سفید در سند بسیار مفید است چون موجب راحت تر شدن خواندن می شود
			\item برای نشان دادن معادل خارجی واژه های فارسی می توان به دو طریق عمل کرد یا در پاورقی نوشته شوند و یا در داخل پرانتز پس از واژه
			\item مراجع مطالبی که در متن گزارش می آید و جزو دانش عمومی نیست باید در داخل یک کروشه مشخص شوند. در متن می توان به نام مولف اشاره داشت یا نداشت، اما نباید از کروشه مرجع به عنوان یک جز از جمله استفاده کرد
		\end{itemize}
	\end{frame}		

	\begin{frame}[plain]{\frametitlefontsize  اجزاء معمول در گزارش های مهندسی - لیست مراجع}
		\framefontsizelarge
		لیست مراجع پس از فصل نتیجه گیری و پیشنهاد ها و قبل از پیوست ها آورده می شود
		
		اطلاعات مراجعی که در متن گزارش به آنها ارجاع داده شده است را ارائه می دهد
		
		مراجع باید به سبک استاندارد نوشته شوند و اطلاعات همه آنها باید کامل باشد
	\end{frame}	
	
	\begin{frame}[plain]{\frametitlefontsize  اجزاء معمول در گزارش های مهندسی - پیوست}
		\framefontsizelarge
		یک صفحه با عنوان ضمائم یا پیوست ها قرار می گیرد و پس از آن ضمائم یکی پس از دیگری آورده می شوند.هر ضمیمه یک شماره حرفی یا عددی و یک عنوان دارد
	\end{frame}	
	
	\begin{frame}[plain]{\frametitlefontsize  اجزاء معمول در گزارش های مهندسی - پیوست}
		\framefontsizelarge
		مطالب آورده شده در ضمیمه ها 
		
		\begin{itemize}\rtlist
			\item مطالبی که حجیم بوده و جریان فکری گزارش را قطع می کند
			\item مطالبی که از نظر نویسنده مطلب اصلی نیست، اما ارائه آنها در گزارش لازم است
			\item داده های زیاد
			\item فلوچارت ها و نقشه های بزرگ و زیاد
			\item لیست برنامه های کامپیوتری
			\item فهرست علائم اختصاری
			\item واژه نامه
		\end{itemize}
	\end{frame}	
	
	\section{\sectionfontsize ارجاع در متن}	
	
	\begin{frame}[plain]{\frametitlefontsize  ارجاع در متن}
		\framefontsizelarge
		ارجاع در متن ممکن است به دو روش انجام شود: با شماره مرجع در فهرست مراجع یا با نام مولف و تاریخ انتشار. 
		
		برای ارجاع در متون انگلیسی در روش  \textbf{انجمن مهندسین برق و الکترونیک امریکا} به مراجع با شماره ارجاع می شود. در این سبک ارجاع به یک صفحه با حرف p. و به چند صفحه با pp. مشخص می شود
		
		ارجاع در متن در روش \textbf{هاروارد} با قرار دادن نام مولفین و تاریخ انتشار مرجع در یک پرانتز انجام می شود. اگر نام نویسنده مستقیما در متن بیاید، پس از نام کافی است که تاریخ انتشار در پرانتز جلوی آن ظاهر شود
		
		در متون فارسی نیز می توانیم همین طور عمل کنیم (به جای حرف p از حرف ص استفاده می کنیم)
	\end{frame}	
	
	\section{\sectionfontsize سبک نگارش مراجع}	
	
	\begin{frame}[plain]{\frametitlefontsize  سبک نگارش مراجع}
		\framefontsizelarge
		روش های بسیار متعددی برای ارائه اطلاعات مراجع در لیست مراجع در انتهای گزارش ها بکار می رود
		
		در لیست مراجع نیز مانند ارجاع در متن به دو روش عمل می کنیم، شماره گذاری یا به ترتیب الفبائی نام خانوادگی مولف اول مرجع و با دو روش ارجاع در متن مطابقت دارند 
		
		روش سوم این است که مراجع بر اساس ترتیب الفبائی مرتب شوند اما به هر مرجع یک شماره نیز تعلق گیرد. این شماره با [1] شروع می شود و به ترتیب صعودی افزایش می یابد. در این حال ارجاع در متن با شماره انجام می شود
	\end{frame}	
	
	\begin{frame}[plain]{\frametitlefontsize  سبک نگارش مراجع}
		\framefontsizelarge
		 گرچه نحوه ارائه اطلاعات مربوط به هر مرجع در لیست مراجع دارای یک روش استاندارد یگانه نیست، اما پژوهشگران نباید تصور کنند که می توانند ارائه اطلاعات را به سلیقه خود تنظیم کنند، بلکه باید از یک روش استاندارد مناسب برای این منظور استفاده نمایند
		 
		 در اینجا اهمیت استفاده از یک نرم افزار مدیریت منابع مانند نرم افزار \textbf{اندنوت} در کنار نرم افزار ویراستاری مانند \textbf{ورد} آشکار می شود
		 
		 دو سبک فارسی که مشابه انجمن مهندسین برق و الکترونیک امریکا و هاروارد می باشند، برای نگارش مراجع فارسی پیشنهاد می شوند
	\end{frame}	
	
	\begin{frame}[allowframebreaks]{\frametitlefontsize  سبک نگارش مراجع - سبک اول}
		\framefontsizelarge
		این سبک بر اساس انجمن مهندسین برق و الکترونیک امریکا با طرح موارد مهمتر و حذف بعضی موارد کم اهمیت تر تهیه شده است.
		
		\begin{itemize}\rtlist
			\item کتاب
			\item یک فصل در کتاب
			\item کتاب بدون مولف
			\item کتاب ترجمه ای
			\item  مقالات در نشریات علمی
			\item مقاله کنفرانس
			\item پایان نامه و رساله
			\item گزارش فنی
			\item پروانه اختراع
			\item کارهای چاپ نشده
			\item استاندارد
			\item مواد درسی
			\item منابع اینترنتی
			\begin{itemize}\rtlist
				\item کتاب
				\item مقالات نشریات علمی
				\item مقالات کنفرانس
				\item گزارش
				\item صفحه وب
				\item پروانه اختراع 
			\end{itemize}
		\end{itemize}
	\end{frame}	

	\begin{frame}[allowframebreaks]{\frametitlefontsize  سبک نگارش مراجع - سبک دوم}
		\framefontsizelarge
		این سبک بر اساس روش هاوارد تهیه شده است
		\begin{itemize}\rtlist
			\item کتاب تالیفی
			\item کتاب گردآوری
			\item کتاب با بیش از چهار نویسنده
			\item یک فصل در کتاب
			\item  کتاب بدون مولف/موسسه مولف
			\item مقاله نشریات علمی
			\item  مقاله علمی از پایگاه های اطلاعاتی
			\item مقاله کنفرانس
			\item پایان نامه و رساله
			\item گزارش های فنی/سالانه
			\item پروانه ثبت اختراع
			\item کارهای چاپ نشده
			\item استاندارد
			\item وب سایت
			\item مواد درسی
			\item مستندات اینترنتی و فایل های پی دی اف
		\end{itemize}
	\end{frame}	

\end{persian}
\end{document}