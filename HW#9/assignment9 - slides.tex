\documentclass[14pt]{beamer}

\mode<presentation>
\usetheme[progressbar=frametitle]{metropolis}

\usepackage{booktabs}
\usepackage[scale=2]{ccicons}
\usepackage{xspace}
\usepackage{xepersian}

\defbeamertemplate*{title page}{customized}[1][]
{	
	\usebeamercolor[fg]{titlegraphic}\inserttitlegraphic\par
	\raggedleft\usebeamerfont{title}\inserttitle\par
	\usebeamerfont{subtitle}\usebeamercolor[fg]{subtitle}\insertsubtitle\par
	\bigskip
	\usebeamerfont{author}\insertauthor\par
	\usebeamerfont{institute}\insertinstitute\par
	\usebeamerfont{date}\insertdate\par
}	

\title{نکات ویژه در انواع ارائه نوشتاری - سازماندهی و زمانبندی ارائه گفتاری}
\subtitle{\color{brown} روش تحقیق و گزارش نویسی}
\date{17 فروردین 1395}
\author{سیاوش کاوسی}
\institute{دانشگاه صنعتی امیرکبیر}
\titlegraphic{\hfill\includegraphics[height=1.5cm]{logo}}

\makeatletter
\newcommand{\rtlist}{\raggedleft\rightskip\@totalleftmargin} 
\makeatother
\newcommand{\sectionfontsize}{\fontsize{22pt}{0pt}\selectfont}
\newcommand{\framefontsizelarge}{\fontsize{18pt}{0pt}\selectfont}
\newcommand{\frametitlefontsize}{\fontsize{20pt}{0pt}\selectfont}
\newcommand{\defaultvspace}{\vspace{5mm}}

% xepersian font settings 
\settextfont{Adobe Arabic}\fontsize{12pt}{0pt}

\begin{document}
\begin{persian}
	\maketitle
	\everypar{\rightskip\rightmargin}		
	
	\section{\sectionfontsize نکات ویژه در انواع ارائه نوشتاری}	
	
	\begin{frame}[plain]{\frametitlefontsize نکات ویژه در انواع ارائه نوشتاری}
		\framefontsizelarge
		مطالب مطرح شده در دو درس گذشته مطالب کلی در تهیه گزارش نوشتاری هستند
		
		ارائه های نوشتاری گرچه دارای جنبه های یکسانی هستند اما در مورد هر یک نکات ویژه ای نیز مطرح است که باید به آن ها توجه کرد
		
		انواع ارائه های نوشتاری رایج
		\begin{itemize}\rtlist
			\item سمینار دانشجوئی
			\item مقاله برای نشریات علمی
			\item مقاله برای همایش ها
			\item پایان نامه های تحصیلی
			\item گزارش وضعیت یا پیشرفت طرح پژوهشی
		\end{itemize}
	\end{frame}	
	
	\begin{frame}[plain]{\frametitlefontsize نکات ویژه در انواع ارائه نوشتاری}
		\framefontsizelarge
		\begin{itemize}\rtlist
			\item گزارش پروژه بر اساس روش علمی
			\item گزارش مطالعه امکان سنجی
			\item گزارش ارزیابی و توصیه
			\item گزارش حادثه
			\item گزارش بازدید
			\item گزارش سفر
			\item گزارش یک آزمایش یا بررسی عملی
			\item گزارش کارآموزی
		\end{itemize}
		\persian
		نکات مهم مرتبط با بعضی ارائه های نوشتاری ضروری مانند دستورالعمل ها، مشخصات فنی، حرفه نامه، و نامه های اداری نیز مطرح می گردد
	\end{frame}	

	\section{\sectionfontsize مقاله برای همایش ها}	

	\begin{frame}[plain]{\frametitlefontsize مقاله برای همایش ها}
		\framefontsizelarge
		ساختار مشابه مقالات نشریات علمی دارند، با این تفاوت که مقالات همایش ها معمولا کوتاه تر هستند و جامعیت مقالات نشریات را ندارند
		
		حداکثر طول مقالات همایش ها معمولا چهار، شش، یا هشت صفحه است که باید اکیدا رعایت شود
		
		مقالات همایش ها قاعدتا باید حاوی ایده های جدید و نوآوری باشند
		
		مقالات ارسالی به همایش ها باید هم از نظر ساختاری و شکلی و هم از نظر محتوای مقاله به فرم نهائی بسیار نزدیک باشند زیرا عمدتا زمانی برای اصلاح و داوری مجدد مقاله نیست
	\end{frame}	

\end{persian}
\end{document}