\documentclass[14pt]{beamer}

\mode<presentation>
\usetheme[progressbar=frametitle]{metropolis}

\usepackage{booktabs}
\usepackage[scale=2]{ccicons}
\usepackage{xspace}
\usepackage{xepersian}

\defbeamertemplate*{title page}{customized}[1][]
{	
	\usebeamercolor[fg]{titlegraphic}\inserttitlegraphic\par
	\raggedleft\usebeamerfont{title}\inserttitle\par
	\usebeamerfont{subtitle}\usebeamercolor[fg]{subtitle}\insertsubtitle\par
	\bigskip
	\usebeamerfont{author}\insertauthor\par
	\usebeamerfont{institute}\insertinstitute\par
	\usebeamerfont{date}\insertdate\par
}	

\title{تولید گزارش نوشتاری نهائی}
\subtitle{\color{brown} روش تحقیق و گزارش نویسی}
\date{17 فروردین 1395}
\author{سیاوش کاوسی}
\institute{دانشگاه صنعتی امیرکبیر}
\titlegraphic{\hfill\includegraphics[height=1.5cm]{logo}}

\makeatletter
\newcommand{\rtlist}{\raggedleft\rightskip\@totalleftmargin} 
\makeatother
\newcommand{\sectionfontsize}{\fontsize{22pt}{0pt}\selectfont}
\newcommand{\framefontsizelarge}{\fontsize{18pt}{0pt}\selectfont}
\newcommand{\frametitlefontsize}{\fontsize{20pt}{0pt}\selectfont}
\newcommand{\defaultvspace}{\vspace{5mm}}

% xepersian font settings 
\settextfont{Adobe Arabic}\fontsize{12pt}{0pt}

\begin{document}
\begin{persian}
	\maketitle
	\everypar{\rightskip\rightmargin}		
	
	\begin{frame}[plain]{\frametitlefontsize تولید گزارش نوشتاری نهائی}
		\framefontsizelarge
		پیش نویس اولیه حتی در فرم اصلاح شده اش فاصله زیادی تا یک گزارش نوشتاری مطلوب دارد. اصلاحات محتوائی، ادبی و ساختاری زیادی باید بر روی پیش نویس اعمال شود تا گزارش به شکل مناسب در آید.
	\end{frame}	
	
	\section{\sectionfontsize ایجاد اصلاحات ساختاری و محتوائی}	
	
	\begin{frame}[plain]{\frametitlefontsize ایجاد اصلاحات ساختاری و محتوائی - تنظیم ساختار هر بخش}
		\framefontsizelarge
		محتویات هر بخش باید به گونه ای اصلاح شود که ساختار آن به صورت زیر در آید:
		\begin{itemize}\rtlist
			\item پاراگراف مقدماتی هر بخش که مقدمه ای به بحث اصلی بخش می دهد
			\item پاراگراف (های) مفاهیم مبنائی که تعاریف، مفاهیم و اصطلاحات مورد نیاز را در یک یا چند پاراگراف شرح می دهند
			\item پاراگراف های اصلی
			\item پاراگراف جمع بندی که مطالب را جمع بندی میکند و ورود به بخش بعد را تسهیل میکند
		\end{itemize}
	\end{frame}	
	
	\begin{frame}[plain]{\frametitlefontsize ایجاد اصلاحات ساختاری و محتوائی - تنظیم ساختار هر بخش}
		\framefontsizelarge
		محتویات هر بخش باید به گونه ای اصلاح شود که ساختار آن به صورت زیر در آید:
		\begin{itemize}\rtlist
			\item پاراگراف مقدماتی هر بخش که مقدمه ای به بحث اصلی بخش می دهد
			\item پاراگراف (های) مفاهیم مبنائی که تعاریف، مفاهیم و اصطلاحات مورد نیاز را در یک یا چند پاراگراف شرح می دهند
			\item پاراگراف های اصلی
			\item پاراگراف جمع بندی که مطالب را جمع بندی میکند و ورود به بخش بعد را تسهیل میکند
		\end{itemize}
	\end{frame}	
	
	\begin{frame}[plain]{\frametitlefontsize ایجاد اصلاحات ساختاری و محتوائی - تنظیم ساختار هر بخش}
		\framefontsizelarge
		در پاراگراف های اصلی، هر کجا که مناسب باشد، ارائه مطالب به صورت لیستی از اقلام که با اعداد یا سمبل نشانه دار می شوند آورده شوند. این کار به درک سریع تر مطالب کمک زیادی می کند
		
		اگر اقلام ترتیب خاصی ندارند، بهتر است از علائم استفاده شود در غیر اینصورت اعداد توصیه می شوند
		
		اگر اقلام جمله ای کامل اند باید در انتهای آن نقطه ای گذاشت
		
		همچنین مطلوب است که توازی دستور زبانی برقرار باشد
	\end{frame}	
	
	\begin{frame}[plain]{\frametitlefontsize ایجاد اصلاحات ساختاری و محتوائی - تنظیم ساختار هر بخش}
		\framefontsizelarge
		در اینجا بهتر است که پاراگراف تعریف شود
		
		یک پاراگراف حاوی یک ایده واحد و مشخص است
		
		ایده اصلی معمولا یا اولین و یا آخرین جمله است که در متون مهندسی بهتر است جمله اصلی اول بیان شود و سپس جزئیات بیاید
		
		اندازه متعارف پاراگراف 100 الی 150 کلمه است و توصیه می شود از 12 خط بیشتر نشود
	\end{frame}	
	
	\begin{frame}[plain]{\frametitlefontsize ایجاد اصلاحات ساختاری و محتوائی - تنظیم طول کل ارائه}
		\framefontsizelarge
		باید طول ارائه بطور متناسب با ماهیت گزارش تنظیم شود
		
		طول متداول برای گزارش 
		\begin{itemize}\rtlist
			\item مقاله کنفرانس: 4 الی 10 صفحه و عموما 15 صفحه
			\item مقاله مجله علمی: 10 الی 25 صفحه و عموما 15 صفحه
			\item پایان نامه تحصیلی: 80 صفحه به بالا
		\end{itemize}
		
		با توجه به نقش هر بخش و متناسب با اهمیت آن باید طول هر بخش و در نتیجه طول کل ارائه را تنظیم کنیم
	\end{frame}	
	
	\begin{frame}[plain]{\frametitlefontsize ایجاد اصلاحات ساختاری و محتوائی - ایجاد پاورقی های لازم}
		\framefontsizelarge
		در جاهائی که آوردن مطلب ضروری به نظر می رسد، اما در روند بیان مطلب گسستگی ایجاد میکند، باید مطلب به صورت پاورقی بیان شود
		
		پاورقی ها نباید در طول گزارش تکراری باشند
		
		اگر تعداد و طول آنها زیاد بود بهتر است به انتهای گزارش منتقل شوند
	\end{frame}	


	\section{\sectionfontsize اصلاح ادبی - نگارشی گزارش}

	\begin{frame}[plain]{\frametitlefontsize اصلاح ادبی - نگارشی گزارش - استفاده از نشانه های اختصاری}
		\framefontsizelarge
		نکات ریز بسیاری باید در اصلاح ادبی/نگارشی مورد توجه قرار گیرند
		
		نشانه های اختصاری برای پرهیز ار تکرار عبارات طولانی در متن بکار گرفته شود
		
		نشانه ها ممکن است اختصاصی باشند و یا عمومی
		
		باید در حد متعادلی از نشانه های اختصاری استفاده کرد
		
		نشانه ممکن است از ترکیب بعضی از حروف اول یا آخر کلمه یا کلمات عبارت کامل ایجاد شود و یا شاخص ترین کلمه عبارت بعنوان نشانه بکار برده شود
	\end{frame}	

	\begin{frame}[plain]{\frametitlefontsize اصلاح ادبی - نگارشی گزارش - نوشتن عدد، شماره، و فرمول}
		\framefontsizelarge
		نوشتن اعداد باید بر اساس قواعد باشد:
		\begin{itemize}\rtlist
			\item اعداد یک الی بیست با حروف نوشته می شوند
			\item دهگان ها و صدگان ها با حروف نوشته می شوند
			\item بجز دهگان ها و صدگان ها عددهای بین بیست و 99 با ارقام نوشته می شوند
			\item کلمه های \textbf{هزار}، \textbf{میلیون} و \textbf{میلیارد} باحروف نوشته می شوند و بقیه با ارقام
		\end{itemize}
	\end{frame}	
	
	\begin{frame}[plain]{\frametitlefontsize اصلاح ادبی - نگارشی گزارش - نوشتن عدد، شماره، و فرمول}
		\framefontsizelarge
		\begin{itemize}\rtlist
			\item اگر اعدادی که باید با حروف نوشته شوند و اعدادی که باید با ارقام نوشته شوند پشت سر هم بیایند یا در جمله های هم پایه بکار روند، همه با \textbf{ارقام} نوشته می شوند
			\item اعداد ترتیبی کمتر از هزار با حروف نوشته می شوند. در اعداد ترتیبی بزرگتر از هزار ارقام بعد از هزار با حروف نوشته می شوند
		\end{itemize}
	\end{frame}	
	
	\begin{frame}[plain]{\frametitlefontsize اصلاح ادبی - نگارشی گزارش - نوشتن عدد، شماره، و فرمول}
		\framefontsizelarge
		شماره بخش ها، شکل ها و جدول ها همه با ارقام نوشته می شوند و در شماره های ترکیبی ارقام با خط تیره از یکدیگر جدا می شوند
		
		نوشتن فرمول ها بر اساس قواعد زیر است:
		\begin{itemize}\rtlist
			\item در بالا و پائین فرمول فاصله اضافی قرار داده شود
			\item شروع فرمول در فاصله یک سانتیمتری از حاشیه سمت چپ متن باشد
			\item شماره فرمول در همان سطر فرمول در فاصله یک سانتیمتری از حاشیه سمت راست متن قرار داده شود. شماره فرمول در پرانتز قرار می گیرد و نیازی به استفاده از کلماتی نظیر رابطه، فرمول، یا معادله در کنار آن نیست
		\end{itemize}
	\end{frame}	
	
	\begin{frame}[plain]{\frametitlefontsize اصلاح ادبی - نگارشی گزارش - نوشتن کلمات خارجی}
		\framefontsizelarge
		در یک گزارش فارسی باید بیشترین تلاش بشود تا کلمات فارسی بکار روند
		
		توصیه در مواقع مواجهه با کلمات خارجی
		\begin{itemize}\rtlist
			\item اسامی خارجی به فارسی نوشته شوند. معادل خارجی آنها را می توان اولین بار در پرانتز در کنار آنها و یا در پاورقی آورد
			\item نام های خارجی دو جز معنی دار را باید جدا نوشت
			\item S $\leftarrow$ اس
			\item در نوشتن معادل فارسی کلمات خارجی که کلمات فارسی مصطلح تثبیت شده ندارند باید سعی شود از پیشوندها و پسوندهای فارسی استفاده شود
		\end{itemize}
	\end{frame}	

	\begin{frame}[plain]{\frametitlefontsize اصلاح ادبی - نگارشی گزارش - درست نویسی در فارسی}
		\framefontsizelarge
		عدم آشنائی با قواعد فارسی و بی توجهی به استفاده از کلمات صحیح منجر به رایج شدن بعضی اشتباهات در نوشته های امروزی شده است
		خلاصه ای از نکات قابل توجه در این زمینه
		\begin{itemize}\rtlist
			\item استفاده از کلمات مرکب به صورت پیوسته در محل مناسب آن - چنانچه در موردی یقین حاصل نشد بهتر است کلمات مرکب جدا نوشته شوند
			\item همزه مفتوح پس از حرف ساکن در وسط کلمه به صورت \textbf{أ} نوشته می شود
			\item کلمات فارسی را نباید با تنوین بکار ببریم
			\item برای جمع کلمات از \textbf{ها} استفاده شود نه \textbf{ات}
		\end{itemize}
	\end{frame}	
	
	\begin{frame}[plain]{\frametitlefontsize اصلاح ادبی - نگارشی گزارش - درست نویسی در فارسی}
		\framefontsizelarge
		\begin{itemize}\rtlist
			\item حرف \textbf{را} باید بدون فاصله بعد از مفعول و وابسته های آن بیاید و باید مراقب تکرار زائد آن بود
			\item حذف کلمه ها یا عبارت های مشترک در دو یا چند جمله برای اجتناب از تکرار خوب است، اما بشرط آنکه نقش دستوری کلمه ها یا عبارت ها یکی باشد
			\item فعل مجهول با وجود فاعل در جمله بکار برده نشود
			\item بین اجزا فعل مرکب فاصله نیفتد
			\item از بکار بردن ضمائر در وضعیت های مبهم پرهیز شود
			\item از توصیف شخصی و بکار بردن افعال شخصی در حد امکان پرهیز شود
		\end{itemize}
	\end{frame}	
	
	\begin{frame}[plain]{\frametitlefontsize اصلاح ادبی - نگارشی گزارش - درست نویسی در فارسی}
		\framefontsizelarge
		\begin{itemize}\rtlist
			\item سبک نگارش محاوره ای نباشد
			\item نگارش در حد امکان ساده باشد
			\item استفاده غلط از کلمات در بعضی موارد رایج شده است
			\item جملات مرکب و طولانی به چند جمله تغییر داده شوند
		\end{itemize}
	\end{frame}	
	
	\begin{frame}[plain]{\frametitlefontsize اصلاح ادبی - نگارشی گزارش - نشانه گذاری}
		\framefontsizelarge
		هدف از نشانه گذاری تسهیل خواندن و درک نوشته ها است. لذا استفاده از نشانه ها در یک متن یک ضرورت است که در کیفیت گزارش تاثیر قابل توجهی دارد
		
		توصیه می شود بجای دخالت دادن لحن گفتار در نشانه گذاری، از قواعد موجود برای نشانه گذاری استفاده شود تا عموم خوانندگان بتوانند گزارش را راحت تر و سریعتر درک کنند
	\end{frame}	
	
	\section{\sectionfontsize افزودن سایر اجزا لازم}
	
	\begin{frame}[plain]{\frametitlefontsize افزودن سایر اجزا لازم}
		\framefontsizelarge
		با انجام اصلاحات در پیش نویس، گزارش اینک به وضعیت مطلوبی رسیده است
		
		باید اجزای دیگری که برای گزارش ضروری است به آن اضافه شود
		
		با افزودن این اجزا گزارش کامل است و تنها باید به شکل نهائی درآورده شود
	\end{frame}	
	
	\section{\sectionfontsize تایپ و نظر خواهی}
	
	\begin{frame}[plain]{\frametitlefontsize تایپ و نظر خواهی}
		\framefontsizelarge
		در صورتی که زمان باقیمانده چندان زیاد نباشد و گزارش طولانی باشد، توصیه می شود که تایپ گزارش به یک تاپیست حرفه ای واگذار شود و فقط اصلاحات توسط شخص انجام شود
		
		گزارش حاوی غلط های تایپی تاثیرات منفی در مخاطبین می گذارد
		
		توصیه می شود افراد دیگر علاوه بر خود ارائه دهنده، گزارش را مرور و اصلاح نمایند
		
		پس از تایپ و اصلاح گزارش، باید گزارش را به افراد مناسبی داد تا آن را مطالعه کنند و در مورد آن اظهارنظر کنند
		
		با بررسی نظرات آنان و تاثیر نظرات در گزارش، باید آخرین ویرایش ها روی گزارش انجام پذیرد
	\end{frame}
	
	\section{\sectionfontsize ویرایش و ارزیابی نهایی}
	
	\begin{frame}[plain]{\frametitlefontsize ویرایش و ارزیابی نهایی}
		\framefontsizelarge
		در آخرین مرحله، ویرایش و ارزیابی نهائی از گزارش انجام و آخرین اصلاحات روی آن اعمال می گردد
		
		بسیاری از نویسندگان بجای خواندن نوشته برای چند بار و پیدا کردن اشکالات بطور اتفاقی، از روش نظام
		مندتری استفاده می کنند که این روش ابتدا صحت فنی مطالب را کنترل می کند و سپس ویرایش
		غیرمحتوائی را در سه سطح کل سند حداقل یک بار مرور می شود:
		\begin{itemize}\rtlist
			\item سطح اول: قالب کلی، سازمان و ظاهر نوشتار کنترل می شود
			\item سطح دوم: به مواردی مانند ساختار و طول جملات و پاراگراف ها و استفاده از کلمات مناسب می پردازد
			\item سطح سوم: نکات ریزی مانند املای صحیح کلمات، علامت گذاری و مانند آن کنترل می شود
		\end{itemize}
	\end{frame}

\end{persian}
\end{document}